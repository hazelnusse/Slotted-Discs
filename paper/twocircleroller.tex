\documentclass{rspublic}

% Custom commands for a few commonly used things
\renewcommand{\b}[1]{\mathbf{#1}}  % For making vector symbols bold
\newcommand{\ra}{$\Longrightarrow$}  % For making a right pointed arrow

\begin{document}

\title[Two Circle Roller]{Dynamics of The Two Circle Roller}

\author[D.L. Peterson, G. Gede, M. Hubbard]{Dale L. Peterson, Gilbert Gede,
Mont Hubbard}

\affiliation{Department of Mechanical and Aerospace Engineering\\ University of
California, 2132 Bainer Hall, One Shields Avenue, Davis, CA 95616, USA}

\label{firstpage}
\maketitle

\begin{abstract}{rolling, dynamics, indeterminate forces}

The 'two circle roller' or 'slotted discs' is a simple device that exhibits
interesting rolling behavior and presents interesting modeling challenges.  It
is comprised of two rigid thin discs, whose planes of symmetry make a non-zero
angle between each other and whose centers are offset in the radial direction
of each disc.  This deceivingly simple system possesses two holonomic
constraints and three integrable motion constraints which result in one degree
of freedom, a rolling motion about the line connecting the contact points of
each disc.  We analyze the particular case when the discs are of equal radii
and whose planes are at a right angle by examining the effect of changing the
ratio of offset distance to disc radii.  Equations of motion are presented,
along with simulation results, and a stability analysis of the equilibrium
points.  The dynamic equations are presented in a dimensionless form and a
bifurcation analysis is presented.

\end{abstract}

\section{Desired plots}
asymmetric configuration mass center height vs. gamma
simulation plot(s):  
\begin{itemize}
 \item Plot of one period oscillations, scaled by the initial
\end{itemize}


\section{Stability of equilibrium}
Two equilibrium configurations: symmetric and asymmetric. 

Symmetric configuration:  at low l's eigenvalues are purely imaginary and
stable (although we can't conclude this from the linear analysis), and as l
increase to about 1.7 (for r = 0.1), we found that the eigenvalues become a
real, equal and opposite pair, implying instability.

Desired plot: eigenvalues of each equilibrium as a function of gamma, using a
normalized value of r=1.0, m=1.0, alpha=pi/2.

Large gamma \ra large separation \\
Small gamma \ra  small separation \\

Define $\gamma = l / (\sqrt{2}r)$

This guy wrote about it \cite{Stewart1966}

\bibliographystyle{rspublicnat}
\bibliography{references}

\end{document}
